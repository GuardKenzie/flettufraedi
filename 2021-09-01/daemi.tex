%vim:ft=tex:
%
\documentclass[a4paper,notitlepage]{article}
\usepackage[utf8]{inputenc}
\usepackage[T1]{fontenc}
\usepackage[icelandic]{babel}
\usepackage{fancyhdr}
\usepackage{amsmath}
\usepackage{amssymb}
\usepackage{enumitem}
\usepackage{mdframed}
\usepackage{amsthm}
\usepackage{tikz}
\usepackage{import}
\usepackage{pdfpages}
\usepackage{transparent}
\usepackage{xifthen}
\usepackage{float}
\usepackage{graphicx}
\usepackage{pdfpages}
\usepackage{transparent}
\usepackage[margin=1.5in]{geometry}

\newcommand{\incfig}[1]{%
    \def\svgwidth{\columnwidth}
    \import{./figures/}{#1.pdf_tex}
}

\fancypagestyle{firstpage}{
    \fancyhf{}
    \fancyhead[L]{Fléttufræði \\ 1. september 2021}
    \fancyhead[R]{Tristan Ferrua Edwardsson \\ tfe1@hi.is}
}

\title{Dæmatími}
\date{}
\author{}

\theoremstyle{plain}
\newmdtheoremenv[innerbottommargin=\topskip]{setning}{Setning}
\theoremstyle{definition}
\newmdtheoremenv[innerbottommargin=\topskip]{skilgr}{Skilgreining}

\newmdenv[
  topline=false,
  bottomline=false,
  skipabove=\topsep,
  skipbelow=\topsep
]{siderules}

\fancypagestyle{otherpages}{
    \fancyhf{}
    \fancyhead[L]{\leftmark}
    \fancyhead[R]{\thepage}
}

\newcommand{\quotes}[1]{,,#1''}

\makeatletter
\newcommand{\leqnomode}{\tagsleft@true}
\newcommand{\reqnomode}{\tagsleft@false}
\makeatother

\renewcommand{\labelenumi}{\alph{enumi}.}

\setlength{\parindent}{0ex}

\pagestyle{otherpages}

\DeclareMathOperator*{\adj}{adj}
\DeclareMathOperator*{\des}{des}
\DeclareMathOperator*{\nul}{Null}
\DeclareMathOperator*{\col}{Col}
\newcommand{\detmat}[1]{\left|\begin{matrix} #1 \end{matrix}\right|}
\newcommand{\vect}[1]{\left[\begin{array}{r} #1 \end{array}\right]}

\begin{document}

\maketitle
\thispagestyle{firstpage}

\section*{Dæmi 2}
\paragraph{Lýsing:}
Let $f_k(n)$ be the number of partitions of $n$ into $k$ distinct parts. Give a combinatorial proof that
\begin{equation*}
    f_k\left(n + \binom{k}{2}\right) = p_k(n)
\end{equation*}

\paragraph{Lausn:}
\begin{enumerate}
    \item We have
        \begin{equation*}
            P(x) = \prod_{j\geq 1} \frac{1}{1-x^j} = \prod_{j \geq 1} (1+x^j + x^{2j} + x^{3j}+ \cdots)
        \end{equation*}

        Here, $j$ is kind of the size of the part and
        \begin{equation*}
            Q(x) = \prod_{j\geq 2} \frac{1}{1-x^j}
        \end{equation*}

        But then
        \begin{equation*}
            P(x) = \frac{1}{1-x}Q(x) \quad \text{or}\quad Q(x) = (1-x)P(x)
        \end{equation*}

    \item From $Q(x) = (1-x)P(x)$ we have
        \begin{align*}
            \sum_{n \geq 0} q(n)x^n &= (1-x)\sum_{n \geq 0}p(n)x^n \\
            &= \sum_{n \geq 0}p(n)x^n - x \sum_{n \geq 0}p(n)x^n
        \end{align*}

        However we have
        \begin{equation*}
            x\sum_{n\geq 0}p(n)x^n = \sum_{n\geq 0}p(n)x^{n+1} = \sum_{n\geq 1} p(n-1)x^n
        \end{equation*}
        so we get
        \begin{equation*}
            \sum_{n \geq 0}p(n)x^n - x \sum_{n \geq 0}p(n)x^n = 1 + \sum_{n \geq 1}(p(n)-p(n-1))x^n
        \end{equation*}
        We can now equate the coefficients, that is
        \begin{equation*}
            q(n) = p(n) - p(n-1),\quad n\geq 1
        \end{equation*}

        \begin{center}
            \rule{25em}{0.5pt}
        \end{center}

        For the combinatorial proof, we first rewrite our expression as
        \begin{equation*}
            q(n) + p(n-1) = p(n)
        \end{equation*}
        
        We notice that $\{(\lambda_1,\dots,\lambda_k) \vdash n \}$ is the disjoint union
        \begin{align*}
            \text{(i)}\quad\{(\lambda_1,\dots,\lambda_k) \vdash n \mid \lambda_k \neq 1\} \cup \{(\lambda_1,\dots,\lambda_k)\vdash n \mid \lambda_k = 1\}\quad\text{(ii)}
        \end{align*}

        The cardinality of (1) is $q(n)$ by definition and the cardinality of (ii) is $p(n-1)$ by removing $\lambda_k = 1$.

\end{enumerate}


\section*{Dæmi 5}
\paragraph{Lýsing} Let $S_n$ be the set of permutations of $[n]$. A \emph{descent} of $\pi = a_1a_2\dots a_n \in S_n$ is an index $i$ such that $a_i > a_{i+1}$.
\paragraph{Lausn:}
\begin{enumerate}
    \item We have
        \begin{equation*}
            f(n) = f(n-1) - 1 + f(n-1) + n - 1 = n + 2(f(n-1) - 1)
        \end{equation*}

    \item We have
        \leqnomode
        \begin{align*}
            \sum_{n\geq 1}f(n)x^n &= \sum_{n\geq 1} (2f(n-1)+n - 2)x^n \\
            \tag*{so} F(x) - 1 &=  2\sum_{n\geq 1} f(n-1)x^n + \sum_{n\geq 1}nx^n - 2\sum_{n\geq 1}x^n \\
            &= 2xF(x) + x \sum_{n\geq 1} nx^{n-1} - \frac{2x}{1-x}
        \end{align*}

        Notice that
        \begin{equation*}
            x\sum_{n\geq 1}n x ^{n-1} = \frac{d}{dx}\left( \sum_{n\geq 0}x^n \right) = \frac{d}{dx} \left( \frac{1}{1-x} \right) = \frac{1}{(1-x)^2}
        \end{equation*}
        Then we have
        \leqnomode
        \begin{align*}
            F(x)-1 = 2xF(x) + \frac{1}{(1-x)^2} - \frac{2x}{1-x} \\
            \tag*{so} F(x) &= \frac{1-3x-3x^2}{(1-x)^2(1-2x)}
        \end{align*}

    \item For the closed formula we could use partial-fraction decomposition.
\end{enumerate}

\end{document}
